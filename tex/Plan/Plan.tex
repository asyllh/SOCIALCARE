\documentclass[a4paper]{article}

% Allows separate files to be used in the main file
\usepackage{subfiles}

% Algorithms
\usepackage{algorithm}
\usepackage[noend]{algpseudocode}
\renewcommand{\algorithmicrequire}{\textbf{Input:}}
\renewcommand{\algorithmicensure}{\textbf{Output:}}
\newcommand{\algorithmicbreak}{\State \textbf{break}}
\newcommand{\Break}{\algorithmicbreak}
\newcommand{\algorithmiccont}{\State \textbf{continue}}
\newcommand{\Continue}{\algorithmiccont}
\renewcommand{\algorithmicreturn}{\State \textbf{return}}
\newcommand{\algorithmicto}{\textbf{ to }}
\newcommand{\To}{\algorithmicto}
\newcommand{\algorithmicrun}{\State \textbf{call }}
\newcommand{\Run}{\algorithmicrun}
\newcommand{\algorithmicoutput}{\textbf{output }}
\newcommand{\Output}{\algorithmicoutput}
\newcommand{\algorithmictrue}{\textbf{true}}
\newcommand{\True}{\algorithmictrue}
\newcommand{\algorithmicfalse}{\textbf{false}}
\newcommand{\False}{\algorithmicfalse}
\newcommand{\algorithmicand}{\textbf{ and }}
\newcommand{\AAnd}{\algorithmicand}
\newcommand{\algorithmicnot}{\textbf{not}}
\newcommand{\Not}{\algorithmicnot}
\newcommand{\algorithmicor}{\textbf{ or }}
\newcommand{\Or}{\algorithmicor}

% Contains advanced math extensions
\usepackage{amsmath}

% Introduces the *proof* environment and the \theoremstyle command
\usepackage{amsthm}

% Adds new symbols to be used in math mode, e.g. \mathbb
\usepackage{amssymb}

% To declare multiple authors
%\usepackage{authblk}

% Provides extra comands as well as optimisation for producing tables
\usepackage{booktabs}
\newcommand{\ra}[1]{\renewcommand{\arraystretch}{#1}}

% Allows customisation of appearance and placements for figures/tables etc.
\usepackage{caption}

\usepackage{comment}

% Adds support for arbitrarily-deep nested lists
\usepackage[inline]{enumitem}

%Support for changing size of font in table footnotes
%\usepackage{etoolbox}

% Improves the interface for defining floating objects such as figures/tables
\usepackage{float}

%\usepackage{fullpage}

% For easy management of document margins and the document page size
% \usepackage[bottom=40mm, right=25mm, left=25mm]{geometry}

% Allows insertion of graphic files within a document
\usepackage{graphicx}

% Manage links within the document or to any URL when you compile in PDF
\usepackage[colorlinks]{hyperref} 
\usepackage[dvipsnames]{xcolor}
%Tikz colours, used in tikz figures only
\colorlet{tBlue}{RoyalBlue!35!Cerulean} %tikz color
\colorlet{tRed}{Red} %tikz color
\definecolor{tGreen}{HTML}{569909} %tikz color
\definecolor{tOrange}{HTML}{FA7602} %tikz color
\definecolor{tLightGreen1}{HTML}{C1E685} %tikz color
\definecolor{tLightOrange1}{HTML}{FFCD4F} %tikz color
\colorlet{tLightGreen}{LimeGreen!70!OliveGreen!45!White}
\colorlet{tLightOrange}{Dandelion!65!White}
\definecolor{tLightPink}{HTML}{FFD4EB} %tikz color
\definecolor{tLightBlue}{HTML}{CEF0FF} %tikz color

%Text colours
\colorlet{myRed}{Red!50!OrangeRed}
\definecolor{myOrange}{HTML}{FA7602}
\definecolor{myGreen}{HTML}{569909}
\definecolor{myAqua}{HTML}{00B1BA} %02BEB8
\definecolor{myBlue}{HTML}{0095FF} %00B3FF
\colorlet{myPurple}{Orchid} 
\colorlet{myPink}{Rhodamine!65!Lavender}
\colorlet{myGray}{Gray!90!White}
%\colorlet{myYellow}{Goldenrod}
\definecolor{myYellow}{HTML}{F9E22E} %FFE72F

%\hypersetup{
%		linkcolor=myPurple,
%		citecolor=myAqua,
%		urlcolor=black
%}

% The document class `elsarticle` uses the \AtBeginDocument comman to define the colour of all link types as blue, so we use the \AtBeginDocument command again to overwrite the colour settings to the colors we choose. REMEMBER TO REMOVE THIS BEFORE SUBMITTING.
%\AtBeginDocument{% 
%	\hypersetup{
%		linkcolor=myPurple,
%		citecolor=myAqua,
%		urlcolor=black
%	}
%} %end of \AtBeginDocument

\newcommand{\intro}[1]{{\color{blue}#1}}

\newcommand{\lit}[1]{{\color{myOrange}#1}} 

\newcommand{\model}[1]{{\color{myBlue}#1}}

\newcommand{\dst}[1]{{\color{myPink}#1}}

\newcommand{\comp}[1]{{\color{myGreen}#1}} 

\newcommand{\conc}[1]{{\color{myPurple}#1}} 

\newcommand{\note}[1]{{\color{myPurple}#1}} 

\newcommand{\alert}[1]{{\color{myRed}#1}}

\newcommand{\chng}[1]{{\color{myRed}#1}}

\newcommand{\done}[1]{{\color{myGray}#1}} % \done{<text>}, makes <text> gray, use to highlight things that are not required or should be ignored.

\newcommand{\ialert}[1]{{\color{myRed}\item#1}} % \ialert{<text>}, makes <text> a red bullet point, use for bullet points of things to do, urgent, or revisit.

\newcommand{\idone}[1]{{\color{myGray}\item#1}} % \idone{<text>}, makes <text> a gray bullet point, use for bullet points that have been completed, that are not required or should be ignored.

\renewcommand{\idone}[1]{} % hides all \idone bullet points

\usepackage{longtable}

% Successor of amsmath
\usepackage{mathtools}

\usepackage{multirow}

% No indentation, space between paragraphs
%\usepackage{parskip}

\usepackage[round]{natbib}

%Include standalone .tex files
\usepackage{standalone}

% Define multiple floats (figures/tables) within one environment with individual captions 1a, 1b etc
\usepackage{subcaption}

%\usepackage[caption=false,font=footnotesize]{subfig}

\usepackage{tabularx}

\usepackage{threeparttable}
%\appto\TPTnoteSettings{\footnotesize} % Change font size of table footnotes to footnotesize

\usepackage{tikz}
\usetikzlibrary{shapes.geometric}
\usetikzlibrary{patterns}

\usepackage{wrapfig}

%% The lineno packages adds line numbers. Start line numbering with
%% \begin{linenumbers}, end it with \end{linenumbers}. Or switch it on
%% for the whole article with \linenumbers after \end{frontmatter}.

\usepackage{lineno}
\usepackage{soul}
\usepackage{textcomp}
\usepackage{eurosym}

\newcommand{\correction}[1]{{\normalsize\textbf{\hl{[correction: #1]}}}}

\newcommand{\ie}{\emph{i.e.}\ }
\newcommand{\eg}{\emph{e.g.}\ }

% Need \usepackage{mathtools} to use floor/ceil below.
% Example: \floor*{\frac{x}{2}}, \ceil*{\frac{x}{2}}
% The asterisk resizes floor/ceil brackets.
\DeclarePairedDelimiter{\floor}{\lfloor}{\rfloor}
\DeclarePairedDelimiter{\ceil}{\lceil}{\rceil}

%Theorem style
%\theoremstyle{plain}% default
\newtheorem{theorem}{Theorem}
%\newtheorem{corollary}{Corollary}
\newtheorem{definition}{Definition}

%\theoremstyle{definition}
%\newtheorem{definition}{Definition}
%\newtheorem{proposition}{Proposition}
%\newtheorem{exmp}{Example}[section]

\begin{document}
\title{Updated Project Plan}
%\author{Asyl L. Hawa}
\date{\today}
\maketitle

\section*{Objectives}
\begin{enumerate}[leftmargin=*]
	\item \textbf{Real Evaluation of the DST}\\
	This will be done over a 3-month period in two ways: (a) comparing the automated plans generated by the DST with plans created manually by staff; and (b) obtaining and assessing reviews from nurses regarding the new automated plans. This will provide us with a clearer understanding of the impact of the DST in practice.
	\item \textbf{Exact Methods for the RSP}\\
	An exact method would provide a guarantee that the problem is solved to optimality, however, it may not be suitable for daily planning due to the time and software that may be required. An exact method would be useful for Health Services for considering whether investments in planning are helpful or necessary, and will also be useful as a basis for other types of modelling such as stochastic modelling (see Objective 3).
	\item \textbf{Stochastic Model for Unexpected Changed in Plans}\\
	We will identify particular situations that require modifications to existing plans, e.g. changes in the number of nurses or new patients requiring treatment, and update the model to take these unplanned changes into consideration.
\end{enumerate}	

%\begin{itemize}
%	\item Compare results from original plan with new DST plans at Abicare
%	\item start looking at exact methods
%	\item look at stochastic method
%	\item publish code and paper
%\end{itemize}

\section*{Outputs}
\begin{enumerate}[leftmargin=*]
	\item Release a stable version of the DST along with documentation and examples for the planning of domiciliary care to make the software readily available to health care teams to assist with creating routes and schedules for staff. This will also allow us to obtain feedback from the users and a comparison with previous schedules.
	\item A methodological paper covering findings regarding exact methods (Objective 2) and adaptations for stochastic demand (Objective 3), which will be submitted to a highly-ranked mathematical journal such as the European Journal of Operational Research (EJOR).
	\item A scientific paper for a healthcare journal that will present our work on the efficiency of our DST in a real setting in comparison to business-as-usual planning (Objective 1). 
\end{enumerate}	

%\begin{itemize}
%	\item Release DST
%	\item Second paper with exact method and stochastic demand
%	\item publish paper comparing abicare to a healthcare journal
%	\item Resubmit paper to computers and OR
%\end{itemize}


\section*{Timeline}
Below is a guide for the remainder of the project, and outlines how the above targets will be achieved:

\textbf{18th October 2021 -- 7th November 2021}\\
Finalise initial publication and prepare to submit to journal.

\textbf{8th November 2021 -- 14th November 2021}\\
Begin research on exact methods for the HHCRSP and stochastic methods for changes in schedules.

\textbf{15th November 2021 -- 21st November 2021}\\
Meet with Abicare to discuss the frequency in which the DST should be used and how results should be collected for future review. Prepare the DST for release, and prepare a questionnaire for nurses and planners at Abicare which will be used as a basis for comparison after the DST has been trialed.

\textbf{22nd November 2021 -- 28th November 2021}\\
Complete the documentation and release the DST. Send out the initial questionnaire to nurses and planners at Abicare.

\textbf{29th November 2021 -- 19th December 2021}\\
Continue research for exact and stochastic methods.

\textbf{20th December 2021 -- 26th December 2021}\\
Start creating models for exact and stochastic methods.

\textbf{3rd January 2022 -- 16th January 2022}\\
Implement models for exact and stochastic methods into DST.

\textbf{17th January 2022 -- 23rd January 2022}\\
Start gathering data/creating instances to test models in DST.

\textbf{24th January 2022 -- 6th February 2022}\\
Begin running experiments using the DST and analyse results.

\textbf{31st January 2022 -- 6th February 2022}\\
Fix any issues with DST (debugging) and continue analysing results.

\textbf{7th February 2022 -- 17th April 2022}\\
Work on publications for mathematical and healthcare journals.

\textbf{7th March 2022 -- 13th March 2022}\\
Gather schedules from Abicare using DST, and send another questionnaire to nurses and planners.

\textbf{18th April 2022 -- 24th April 2022}\\
Continue working on publications for mathematical and healthcare journals.

\textbf{25th April 2022 -- 30th April 2022}\\
Conclude project.


\begin{comment}
	\begin{enumerate}[leftmargin=*]
	\item \textbf{18th October 2021 -- 31st October 2021}\\
	Finalise initial publication and prepare to submit to journal.
	\item \textbf{November 2021}\\
	Release the DST. Prepare a questionnaire for nurses and planners in industry, which will be used as a basis for comparison after the DST has been used for a period of time. Meet with Abicare to discuss the frequency in which the DST should be used, and how results should be collected for future review. Complete formal documentation for the DST.
	\item \textbf{December 2021 -- January 2022}\\
	Focus on adapting the DST for stochastic demand. Identify the situations that should be considered, and attempt to model the changes.
	\item \textbf{January 2022 -- February 2022}\\
	Turn attention to exact methods. Study the literature to determine methods that will/not be suitable for our DST, and develop a model that can be adapted for the problem.
	\item \textbf{March 2022}\\
	Obtain results from Abicare. Send out final questionnaire to nurses and planners and compare with initial questionnaire. Compare and analyse outputs.
	\item \textbf{April 2022 -- May 2022}\\
	Gather data and test stochastic demand and exact method changes to the model.
	\item \textbf{June 2022}\\
	Potentially release new version of DST including stochastic demand and exact method. Prepare articles for submission to mathematical and healthcare journals. Conclude project.
	\end{enumerate}
	
	\begin{enumerate}[leftmargin=*]
	\item \textbf{18th October 2021 -- 7th November 2021}\\
	Finalise initial publication and prepare to submit to journal.
	\item \textbf{8th November 2021 -- 14th November 2021}\\
	Begin research on exact methods for the HHCRSP and stochastic methods for changes in schedules.
	\item \textbf{15th November 2021 -- 21st November 2021}\\
	Meet with Abicare to discuss the frequency in which the DST should be used and how results should be collected for future review. Prepare the DST for release, and prepare a questionnaire for nurses and planners at Abicare which will be used as a basis for comparison after the DST has been trialed.
	\item \textbf{22nd November 2021 -- 28th November 2021}\\
	Complete the documentation for the DST, and release the DST. Send out the initial questionnaire to nurses and planners at Abicare.
	\item \textbf{29th November 2021 -- 19th December 2021}\\
	Continue research for exact and stochastic methods.
	\item \textbf{20th December 2021 -- 26th December 2021}\\
	Start creating models for exact and stochastic methods.
	\item \textbf{3rd January 2022 -- 16th January 2022}\\
	Implement models for exact and stochastic methods into DST.
	\item \textbf{17th January 2022 -- 23rd January 2022}\\
	Start gathering data/creating instances to test models in DST.
	\item \textbf{24th January 2022 -- 6th February 2022}\\
	Begin running experiments using the DST and analyse results.
	\item \textbf{31st January 2022 -- 6th February 2022}\\
	Fix any issues with DST (debugging) and continue analysing results.
	\item \textbf{7th February 2022 -- 17th April 2022}\\
	Work on publications for mathematical and healthcare journals.
	\item \textbf{7th March 2022 -- 13th March 2022}\\
	Gather schedules from Abicare using DST, and send another questionnaire to nurses and planners.
	\item \textbf{18th April 2022 -- 24th April 2022}\\
	Continue working on publications for mathematical and healthcare journals.
	\item \textbf{25th April 2022 -- 30th April 2022}\\
	Conclude project.
	\end{enumerate}	
\end{comment}





























\end{document}