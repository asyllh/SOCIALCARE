%---- PREAMBLE BEAMER----%

% Allows separate files to be used in the main file
\usepackage{subfiles}

% Algorithms
\usepackage{algorithm}
\usepackage[noend]{algpseudocode}
\renewcommand{\algorithmicrequire}{\textbf{Input:}}
\renewcommand{\algorithmicensure}{\textbf{Output:}}
\newcommand{\algorithmicbreak}{\State \textbf{break}}
\newcommand{\Break}{\algorithmicbreak}
\renewcommand{\algorithmicreturn}{\State \textbf{return}}
\newcommand{\algorithmicto}{\textbf{ to }}
\newcommand{\To}{\algorithmicto}
\newcommand{\algorithmicrun}{\State \textbf{call }}
\newcommand{\Run}{\algorithmicrun}
\newcommand{\algorithmicoutput}{\textbf{output }}
\newcommand{\Output}{\algorithmicoutput}
\newcommand{\algorithmictrue}{\textbf{true}}
\newcommand{\True}{\algorithmictrue}
\newcommand{\algorithmicfalse}{\textbf{false}}
\newcommand{\False}{\algorithmicfalse}
\newcommand{\algorithmicand}{\textbf{ and }}
\newcommand{\AAnd}{\algorithmicand}
\newcommand{\algorithmicnot}{\textbf{not}}
\newcommand{\Not}{\algorithmicnot}
\newcommand{\algorithmicor}{\textbf{ or }}
\newcommand{\Or}{\algorithmicor}


% Contains advanced math extensions
\usepackage{amsmath}

% Introduces the *proof* environment and the \theoremstyle command
%\usepackage{amsthm}

% Adds new symbols to be used in math mode, e.g. \mathbb
\usepackage{amssymb}

% Provides extra comands as well as optimisation for producing tables
\usepackage{booktabs}
\newcommand{\ra}[1]{\renewcommand{\arraystretch}{#1}}

% Allows customisation of appearance and placements for figures/tables etc.
\usepackage{caption}
%\captionsetup[algorithm]{labelformat=empty}

% Adds support for arbitrarily-deep nested lists
%\usepackage[inline]{enumitem}

% Improves the interface for defining floating objects such as figures/tables
\usepackage{float}

%\usepackage{fullpage}

% For easy management of document margins and the document page size
%\usepackage{geometry}

% Allows insertion of graphic files within a document
\usepackage{graphicx}

% Manage links within the document or to any URL when you compile in PDF
\usepackage{hyperref} 
%\usepackage{xcolor} %Already inbuilt in beamer

%Text colours
\colorlet{myRed2}{Red!50!OrangeRed}
\definecolor{myOrange}{HTML}{FA7602}
\definecolor{myGreen}{HTML}{569909}
\definecolor{myAqua}{HTML}{00B1BA} %02BEB8
\definecolor{myBlue1}{HTML}{0095FF} %00B3FF
\colorlet{myPurple}{Orchid} 
\colorlet{myPink}{Rhodamine!65!Lavender}
\colorlet{myGray}{Gray!90!White}
\definecolor{myYellow}{HTML}{F9E22E} %FFE72F
\definecolor{myLightGreen}{HTML}{A9D438}
\definecolor{myBlue}{HTML}{3588F5}
\definecolor{myBlue2}{HTML}{0F88F1} %00B3FF %0095ff
\colorlet{myBlue3}{RoyalBlue!35!Cerulean} %tikz color
\definecolor{myRed1}{HTML}{F7B5AC}
\definecolor{myRed}{HTML}{FFCCCC}

\newcommand{\red}[1]{{\color{myRed2}#1}}
\newcommand{\gre}[1]{{\color{myGreen}#1}}
\renewcommand{\note}[1]{{\color{myPink}#1}} %\note is already inbuilt in Beamer, so need to \renewcommand

% Successor of amsmath
\usepackage{mathtools}

% For multicolumn itemize lists
\usepackage{multicol}

% No indentation, space between paragraphs
%\usepackage{parskip}

%Include standalone .tex files
\usepackage{standalone}

% Define multiple floats (figures/tables) within one environment with individual captions 1a, 1b etc
\usepackage{subcaption}

% Tables
\usepackage{tabularx}

% Tikz figures
\usepackage{tikz}

% Wrapping figures around text
\usepackage{wrapfig}

% Quote text
\usepackage{verbatim}



\DeclarePairedDelimiter{\floor}{\lfloor}{\rfloor}
\DeclarePairedDelimiter{\ceil}{\lceil}{\rceil}


% Theme:
\usetheme[progressbar=foot, numbering=none]{metropolis} %add numbering=none for no slidenumbers
\setbeamersize{text margin left=7mm, text margin right=7mm} %Left and right margins for slides
\setbeamertemplate{itemize items}{\textbullet}
\setbeamercolor{progress bar}{fg=myBlue, bg=gray}
\setbeamercolor{background canvas}{bg=white}
\setbeamercolor{frametitle}{fg=white, bg=myBlue}
\setbeamercolor{item}{fg=myBlue}
%\setbeamercolor{subitem}{fg = white}
%\setbeameroption{show notes}

\setbeamercolor{block title alerted}{fg=black,bg=myRed}
\setbeamercolor{block body alerted}{fg = black,	bg=myRed!50!white}
\setbeamertemplate{itemize subitem}[triangle]

\addtobeamertemplate{frametitle}{}{\vspace{-2em}} %top margin of slides
\addtobeamertemplate{footline}{}{\vspace{-6em}} %bottom margin of slides

\setbeamerfont{caption}{size=\tiny}
\DeclareCaptionFont{tiny}{\tiny} 
%\captionsetup[sub]{font+=tiny}

\newcommand*\aitem{%
	\item[\textcolor{myRed2}{$\bullet$}]}
\newcommand*\ditem{%
	\item[\textcolor{myGreen}{$\bullet$}]}

\makeatletter
\let\@@magyar@captionfix\relax
\makeatother
